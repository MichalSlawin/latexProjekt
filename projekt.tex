\documentclass{article}
\usepackage{polski}
\usepackage[utf8]{inputenc}
\title{Całki}
\begin{document}
\maketitle
\tableofcontents
\section{Całka}
Ogólne określenie wielu różnych, choć powiązanych ze sobą pojęć analizy matematycznej. Najczęściej przez całkę rozumie się całkę oznaczoną lub całkę nieoznaczoną, choć istnieje wiele innych odmian całki.
\section{Całka nieoznaczona}
Przez całkę nieoznaczoną (albo funkcję pierwotną) rozumie się pojęcie odwrotne do pochodnej funkcji. Całkę oznaczoną na przedziale [a,b] można też zdefiniować (tzw. całka Newtona-Leibniza) jako różnicę między wartościami całki nieoznaczonej w punktach b oraz a. Stąd obliczenie całki nieoznaczonej jest często pierwszym krokiem przy obliczaniu całek oznaczonych.

Uogólnieniem całki nieoznaczonej jest całka równania różniczkowego będąca rozwiązaniem równania różniczkowego: $F'(x)=f(x)$, gdzie $F(x)$ jest pierwotną, $f(x)$ a oznacza całkowaną funkcję.
\section{Całka oznaczona}

\end{document}